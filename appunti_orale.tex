\documentclass[fleqn,12pt]{article}
\usepackage[T1]{fontenc}
\usepackage[italian]{babel}
\usepackage[utf8]{inputenc}
\usepackage{graphicx}
\usepackage{amsmath}
\usepackage{multicol}
\usepackage{geometry}
\geometry{
  a4paper,
  left=1.7cm,
  right=1.7cm,
  top=2cm,
  bottom=3cm,
}

\setlength{\mathindent}{0pt}

\begin{document}
\section{Modelli}
\subsection{Primale standard}
    \begin{gather*}
    \begin{cases}
        \text{max} \ \ c^Tx \\
        Ax \le b
    \end{cases}
    \end{gather*}

\subsection{Duale standard}
    \begin{gather*}
    \begin{cases}
        \text{max} \ \ c^Tx \\
        y^TA=c^T \\
        y \ge 0
    \end{cases}
    \end{gather*}

\subsection{Primale ausiliario}
Data una base non ammissibile $B$, definiamo l'insieme dei vincoli soddisfatti
$U = \{i \in N | A_i\bar{x} \le b_i\}$ e l'insieme dei vincoli non soddisfatti
$V = \{i \in N | A_i\bar{x} > b_I\}$. Il problema ausiliario è: \\
    \begin{gather*}
    \begin{cases}
        \text{max} \ -\sum\limits_{i \in V} \varepsilon_i \\
        A_ix \le b_i & \forall i \in B \cup U \\
        A_ix - \varepsilon_i \le b_i & \forall i \in V \\
        -\varepsilon_i \le 0 & \forall i \in V \\
    \end{cases}
    \end{gather*}

\subsection{Produzione}
    \begin{gather*}
    \begin{cases}
        max \sum\limits_{j=1}^{n} c_jx_j \\
        \sum\limits_{j=1}^n a_{i,j}x_j \le b_i & \forall i \in \{1,...,m\} \\
        x_j \ge 0 & \forall j \in \{1,...,n\}
    \end{cases}
    \end{gather*}
\subsection{Assegnamento di costo minimo} \\
    $x_{i,j} = 1$ indica che la persona $i$ svolge il lavoro $j$.
    \begin{gather*}
    \begin{cases}
        \text{min} \,\, cx \\
        \sum\limits_{j=1}^{n} x_{i,j} = 1 & \forall i \in \{1,...,n\} \\
        \sum\limits_{i=1}^{n} x_{i,j} = 1 & \forall j \in \{1,...,n\} \\
        x \ge 0
    \end{cases}
    \end{gather*}
\subsection{Flusso di costo minimo}
    \begin{gather*}
    \begin{cases}
        \text{min} \,\, \sum\limits_{(i,j) \in A} c_{i,j}x_{i,j} \\
        \sum\limits_{(p,i) \in A} x_{p,i} - \sum\limits_{(i,q) \in A} x_{i,q}
        = b_i & \forall i \in N \\
        l_{i,j} \le x_{i,j} \le u_{i,j} & \forall(i,j) \in A
    \end{cases}
    \end{gather*}
\subsection{Flusso di costo minimo non capacitato}
    \begin{gather*}
    \begin{cases}
        \text{min} \,\, c^Tx \\
        Ex = b \\
        x \ge 0
    \end{cases}
    \end{gather*}
\subsection{Potenziale su reti non capacitate}
    \begin{gather*}
    \begin{cases}
        \text{max} \,\, \pi^Tb \\
        \pi^TE \le c^T
    \end{cases}
    \end{gather*}
\subsection{Flusso massimo}
    \begin{gather*}
    \begin{cases}
        \text{max} \,\, v \\
        Ex = b \\
        0 \le x \le u
    \end{cases}
    \end{gather*}
    con
    \begin{gather*}
    b_i =
        \begin{cases}
            -v & \text{se} \> i = s \\
            v  & \text{se} \> i = t \\
            0  & \text{altrimenti}
        \end{cases}
    \end{gather*}
    e
    $v = \sum\limits_{(s,j) \in A} x_{s,j} = \sum\limits_{(i,t) \in A} x_{i,t}$
\subsection{TSP --- Commesso viaggiatore}
\subsubsection{Asimmetrico}
    \begin{gather*}
    \begin{cases}
        \text{min} \sum\limits_{(i,j) \in A} c_{i,j}x_{i,j} \\
        \sum\limits_{i \in N, i \neq j} x_{i,j} = 1 & \forall j \in N \\
        \sum\limits_{j \in N, j \neq i} x_{i,j} = 1 & \forall i \in N \\
        \sum\limits_{i \in S}\sum\limits_{j \notin S} x_{i,j} \ge 1
            & \forall S \subset N, S \neq \emptyset \\
        x_{i,j} \in \{0,1\} & \forall (i.j) \in A
    \end{cases}
    \end{gather*}
\subsubsection{Simmetrico}
    \begin{gather*}
    \begin{cases}
        \text{min} \sum\limits_{(i,j) \in A} c_{i,j}x_{i,j} \\
        \sum\limits_{(h,i) \in A} x_{h,i} +
            \sum\limits_{(i,k) \in A} x_{i,k} = 2 & \forall i \in N \\
        \sum\limits_{(i,j) \in A, i \in S, j \notin S} x_{i,j} +
            \sum\limits_{(i,j) \in A, i \notin S, j \in S} x_{i,j} \ge 1 &
            \forall S \subset N, 1 \le |S| \le
            \Big{\lceil}\frac{|N|}{2}\Big{\rceil} \\
        x_{i,j} \in \{0,1\} & \forall (i.j) \in A
    \end{cases}
    \end{gather*}
\subsection{Taglio di capacità minima (duale MAX-FLOW)}
    \begin{gather*}
    \begin{cases}
        \text{min} \sum\limits_{(i,j) \in A} u_{i,j} \mu_{i,j} \\
        \pi_i - \pi_j + \mu_{i,j} \ge 0 & \forall (i,j) \in A \\
        \pi_A - \pi_S = 1 \\
        \mu \ge 0
    \end{cases}
    \end{gather*}
\subsection{Zaino}
    \begin{gather*}
    \begin{cases}
        \text{max} \sum\limits_{i=1}^n c_ix_i \\
        \sum\limits_{i=1}^n a_ix_i \le b \\
        \sum\limits_{i=1}^n p_ix_i \le p \\
        x_i \in \{0,1\} & \forall i = 1,...,n
    \end{cases}
    \end{gather*}

\section{Programmazione lineare}
\begin{itemize}
    \item Un poliedro in $\mathbb{R}^n$ è l'intersezione di un numero finito di
        semispazi chiusi di $\mathbb{R}^n$.
    \item \textbf{Combinazione convessa}: un punto $x \in \mathbb{R}^n$ è
        combinazione convessa di $x_1,...,x_m \in \mathbb{R}^n$ se esistono dei
        valori $\lambda_1,...,\lambda_m \in \mathbb(R)$ tali che
        $x = \sum\limits_{i=1}^{m} \lambda_ix_i$.
    \item \textbf{Vertice}: punto non esprimibile come combinazione convessa di
        altri punti del poliedro.
    \item Una direzione di recessione $d$ è un vettore tale che:
        $x+\lambda d \in P$, $\forall x \in P$.
    \item \textbf{Weyl -- rappresentazione dei poliedri}: Dato un poliedro P,
        esistono un sottinsieme finito $v = \{v^1,...,v^m\}$ di P e un insieme
        finito $E = \{e^1,...,e^p\}$ (eventualmente anche vuoti) tali che: \\
        $P = \text{conv}(V)+\text{cono}(E)$.
    \item \textbf{Teorema fondamentale della PL}: Se il problema $(P)$ ha un
        valore ottimo finito allora esiste un vertice $v^k$ del poliedro $P$
        che ha valore ottimo.
    \item \textbf{Dualità forte}: Dato il problema $(P)$ e il suo duale $(D)$,
        se i rispettivi poliedri sono non vuoti allora $-\infty < v(P) = v(D) <
        +\infty$.
    \item \textbf{Scarti complementari}: date $\bar{x}$ e $\bar{y}$ soluzioni
        ammissibili per $(P)$ e $(D)$, sono soluzioni ottime se e solo se
        $\bar{y}(b-A\bar{x}) = 0$. \\
        (Conseguenza della dualità forte, infatti $c^T\bar{x} = \bar{y}^Tb =
        \bar{y}^TA\bar{x}$).
    \item Una soluzione di base è ammissibile se è solo se è un vertice del
        poliedro associato al problema.
    \item Il simplesso è un algoritmo che parte da un vertice del poliedro e si
        sposta ad ogni iterazione su un vertice adiacente (caso non degenere),
        oppure rimane sullo stesso vertice ma con una base diversa nel caso
        degenere.
    \item Regole anticiclo di Bland per simplesso primale: \\
        $h = min\{i | y_i < 0, i \in N\}$ \\
        $k = min\{i | \frac{b_i - A_i\bar{x}}{A_iW^h} > 0, i \in N\} $
    \item Regole anticiclo di Bland per simplesso duale: \\
        $k = \text{min}\{i \in N | b_i - A_i\bar{x} < 0\}$ \\
        $h = \text{min}\{\frac{\bar{y}_i}{-A_kW^i} | -A_kW^i > 0\}$
    \item I problemi ausiliari (standard e duale) servono a trovare una base
        ammissibile partendo da una base non ammissibile e la relativa
        soluzione.
    \item Se il valore ottimo di $(P_{aux}) < 0$ allora $(P)$ non ha soluzioni
        ammissibili (sostituire il < con > nel caso del duale). \\
        Se il valore ottimo di $(P) = 0$ allora esiste una base ammissibile per
        $(P)$ che si costruisce a partire dalla base ottima di $(P_{aux})$.
    \item \textbf{Simplesso primale}
\end{itemize}

\section{Programmazione lineare su reti}
\begin{itemize}
    \item Un albero di copertura su un grafo è un albero composto da $n-1$
        archi che toccano tutti i nodi una sola volta senza creare cicli.
    \item Un sottinsieme di archi di un grafo è una base del problema se è solo
        se è un albero di copertura.
    \item \textbf{Bellman}: dato T un albero di copertura che genera un flusso
        di base ammissibile, se $C_{i,j}^\pi \le 0 \,\, \forall (i,j) \in L$,
        con $C_{i,j}^\pi = \Pi_i-\Pi_j+C_{i,j}$, la soluzione è ottima.
    \item \textbf{Bellman su reti capacitate}: data una tripartizione $T, L, U$
        che genera un flusso di base ammissibile, se $C_{i,j}^\pi \le 0 \,\,
        \forall (i,j) \in L$, e $C_{i,j}^\pi \ge 0 \,\,
        \forall (i,j) \in U$, allora la soluzione è ottima.
    \item \textbf{Taglio}: un taglio è una partizione dei nodi di una rete in
        due insiemi $N_s$ e $N_t$ in modo che $N_s \cup N_t = N$ e $N_s \cap
        N_t = \emptyset$. \\
        Il taglio è \textbf{ammissibile} se $N_s$ contiene almeno l'origine $s$
        e $N_t$ contiene almeno la destinazione $t$.
    \item \textbf{Archi diretti/indiretti}: dato un taglio $(N_s, N_t)$ sono
        definiti gli insiemi di archi diretti e indiretti: \\
        $A^+ = \{(i,j) \in A | i \in N_s, j \in N_t\}$ \\
        $A^- = \{(i,j) \in A | i \in N_t, j \in N_s\}$
    \item \textbf{Capacità del taglio}: $u(N_s, N_t) = \sum\limits_{(i,j) \in
        A^+} u_{i,j}$. \\
        (ovvero la somma delle capacità degli archi diretti)
    \item \textbf{Flusso sul taglio}: $x(N_s, N_t) = \sum\limits_{(i,j) \in
        A^+} x_{i,j} - \sum\limits_{(i,j) \in A^-} x_{i,j}$. \\
        (differenza tra il flusso degli archi diretti e il flusso degli archi
        indiretti)
    \item \textbf{Potenziale di base}: $C_{i,j}^\pi = \Pi_i - \Pi_j + C_{i,j}$
        è ammissibile se tutti i costi ridotti degli archi di L sono positivi e
        tutti i costi ridotti degli archi di U sono negativi. Inoltre è
        degenere se esiste un costo ridotto uguale a 0.
\end{itemize}

\section{Programmazione lineare intera}
\begin{itemize}
    \item \textbf{Equivalenza tra PL e PLi}: dato un problema di PLi esiste un
        problema di PL equivalente che ha come soluzione ottima lo stesso
        valore del problema di PLi.
    \item \textbf{Disuguaglianza valida}: una disuguaglianza del tipo
        $\pi^Tx \le \pi_0$ è detta \textit{valida} se vale $\forall x \in
        \Omega$.
    \item \textbf{Piano di taglio}: disuguaglianza valida per $\Omega$ per cui
        vale $\pi^T\bar{x} > \pi_0$.
    \item \textbf{Piano di Gomory}: piano di taglio dato da: $\sum\limit_{j
        \in N} \{\tilde{a}_{r,j}\}x_j \ge \{\tilde{b}_r\}$.
        Dove $\tilde{A} = A_B^{-1}A_N$ e $\tilde{b}=\bar{x}_B$.
\end{itemize}

\end{document}

