\documentclass[9pt]{extarticle}
\usepackage[T1]{fontenc}
\usepackage[italian]{babel}
\usepackage[utf8]{inputenc}
\usepackage{graphicx}
\usepackage{geometry}
\geometry{
  a4paper,
  left=9mm,
  right=9mm,
  top=9mm,
  bottom=9mm,
}
\pagestyle{empty}
\usepackage{multicol}
\setlength{\columnsep}{9mm}

\begin{document}

\begin{multicols}{2}

\section{Soluzione primale e duale}
\subsection{Primale}
\begin{itemize}
    \setlength{\itemsep}{-3pt}
    \item Risolvi il sottosistema dato dai vincoli di base
    \item La soluzione X è \textbf{ammissibile} se tutti i vincoli sono
        rispettati.
    \item La soluzione X è \textbf{degenere} se almeno un vincolo non di base
        è soddisfatto con l'\textit{uguale}.
\end{itemize}
\subsection{Duale}
\begin{itemize}
    \setlength{\itemsep}{-3pt}
    \item Scrivere il problema in forma duale.
    \item Risolvi il sottosistema dato dalle colonne di base.
    \item La soluzione Y è \textbf{ammissibile} se tutte le componenti
        $Y_i \ge 0$.
    \item La soluzione X è \textbf{degenere} se almeno una componente di base
        $Y_i = 0$.
\end{itemize}

\section{Simplesso primale}
\begin{itemize}
    \setlength{\itemsep}{-3pt}
    \item Data una base si trovano $\bar{x}$ e $\bar{y}$ soluzioni di base
        primale e duale.
    \item Indice uscente: $h = min\{i \in N | Y_i < 0 \}$.
    \item Calcolare $W = -A_b^{-1}$ e $W^h$ la colonna di indice $h$.
    \item Per ogni indice non di base $i$ calcolare
        $R_i = \frac{b_i-A_i\bar{x}}{A_iW^h}$.\\
        (Tip: calcolare prima il denominatore: se è < 0 scartarlo)
    \item Indice entrante: $\Theta = min\{R_i | A_iW^h > 0\}$.
\end{itemize}

\section{Simplesso duale}
\begin{itemize}
    \setlength{\itemsep}{-3pt}
    \item Data una base si trovano $\bar{x}$ e $\bar{y}$ soluzioni di base
        primale e duale.
    \item Indice entrante: $k = min\{i \in N | b_i - A_i \bar{x} < 0 \}$.
    \item Calcolare $W = -A_b^{-1}$ e $W^i$ la colonna di indice $i$.
    \item Per ogni indice non di base $i$ calcolare
        $R_i = \frac{\bar{y}_i}{-A_kW^i}$.\\
        (Tip: calcolare prima il denominatore: se è < 0 scartarlo)
    \item Indice uscente: $h = min\{R_i | -A_iW^h > 0\}$.
\end{itemize}

\section{Soluzione e potenziale di flusso}
\subsection{Soluzione di base}
\begin{itemize}
    \setlength{\itemsep}{-3pt}
    \item Gli archi sono partizionati in T (incognite), U (saturati), L
        (azzerati). \\
        Dedurre i bilanci degli archi in T partendo dalle foglie, e fissando il
        flusso degli archi in U alla loro capacità massima.
    \item La soluzione è \textbf{ammissibile} se tutti gli archi in T sono
        $\ge 0$.
    \item La soluzione è \textbf{degenere} se un arco di T è saturato oppure 0.
\end{itemize}
\subsection{Potenziale di base}
\begin{itemize}
    \setlength{\itemsep}{-3pt}
    \item $\pi_i$ è il costo per raggiungere il nodo $i$ utilizzando gli archi
        di T. Se un arco viene percorso al contrario, il costo è considerato
        negativo. Il primo nodo ha costo 0.
    \item Per ogni arco $(i,j) \notin T$ calcolare $C_{i,j}^\pi = \pi_i - \pi_j
        + C_{i,j}$.
    \item Se esiste almeno un $C_{i,j}^\pi < 0$ con $(i,j) \in L$, oppure se
        esiste almeno un $C_{i,j}^\pi > 0$ con $(i,j) \in U$, la soluzione è
        \textbf{non ammissibile}.
    \item Se esiste almeno un $C_{i,j}^\pi = 0$ con $(i,j) \in L \lor (i,j)\in
        U$, allora la soluzione è \textbf{degenere}.
\end{itemize}

\section{Simplesso su reti}
\begin{itemize}
    \setlength{\itemsep}{-3pt}
    \item Calcolare flusso e potenziale di base partendo dagli archi T e U
        dati.
    \item Arco entrante: \\
        $(p,q) = min(\{(i,j) \in L | C_{i,j}^\pi < 0\} \cup
        \{(i,j) \in U | C_{i,j}^\pi > 0\})$ \\
        (il minimo è riferito all'ordine lessicografico)
    \item Inserito il nuovo arco, questo formerà un ciclo con gli altri archi
        di T. Consideriamo il ciclo in verso concorde a $(p,q)$ se apparteneva
        a L, discorde se apparteneva a U.
    \item Calcolare $\Theta^+ = min\{U_{i,j}-\bar{x}_{i,j} | (i,j) \in C^+\}$\\
        $\Theta^- = min\{\bar{x}_{i,j} | (i,j) \in C^-\}$\\
        Infine $\Theta = min(\Theta^+, \Theta^-)$ (in caso di parità vince
        $\Theta^-$).
    \item Arco uscente: arco associato a $\Theta$.
\end{itemize}

\section{Altri algoritmi su grafi}
\subsection{Cammino di costo minimo --- Dijkstra}
\begin{itemize}
    \setlength{\itemsep}{-3pt}
    \item Si inizia visitando la radice, con tutti i costi pari a $\infty$.
    \item Aggiornare i costi per raggiungere i vari nodi partendo dal nodo
        visitato se inferiori a quelli attuali.
    \item Aggiungere all'insieme Q i nodi diventati raggiungibile. Il prossimo
        nodo da visitare è il più economico in Q.
\end{itemize}
\subsection{Algoritmo FFEK}
\begin{itemize}
    \setlength{\itemsep}{-3pt}
    \item Trovare il cammino aumentante: scrivere la stella dalla radice. I
        nodi estratti in questo modo sono una coda con politica FIFO.
    \item Aggiungere in coda la stessa uscente del prossimo nodo.
    \item Il cammino termina appena si estrae l'ultimo nodo.
    \item Trovare $\delta = min\{r_{i,j} | (r,j) \in Cammino\}$ e aggiornare le
        $\bar{x}_{i,j}$ degli archi coinvolti sommandoci $\delta$.
    \item Aggiornare il grafo residuo sottraendo $\delta$ agli archi coinvolti e
        aggiungendola nel verso opposto.
    \item L'algoritmo termina quando non si riesce a costruire un cammino
        aumentante. I nodi riusciti a visitare formano $N_s$, gli altri $N_t$.
\end{itemize}

\section{Taglio di Gomory}
\begin{itemize}
    \setlength{\itemsep}{-3pt}
    \item Risolvere il rilassato continuo del problema utilizzando il metodo
        grafico (nei problemi di minimo si prende il primo punto di
        intersezione del vincolo con l'asse, nei problemi di massimo si prende
        il secondo).
        \textbf{Nota:} approssima la soluzione trovata (per eccesso nei
        problemi di minimo, per difetto nei problemi di massimo).
    \item L'altro limite si trova approssimando le $\bar{x}_i$.
    \item Scrivere il sistema rendendo i vincoli equazioni e inserendo
        variabili di scarto (negative se problema di min.).
    \item Calcolare le variabili di scarto nella soluzione del RC per capire
        quali indici sono di base, dopodichè $\tilde{A} = A_B^{-1}A_N$.
    \item Scrivere $\sum_{j \in N} \{\tilde{a}_{r,j}\} \bar{x}_j \ge
        \{(X_{RC})_r\}$. Poi sostituire alle variabili di scarto quello che si
        ricava a partire dai vincoli. Alla fine deve rimanere una disequazione
        con parametri $x_1$ e $x_2$.
        (\textbf{Nota:} $\{x\} = x - \lfloor x \rfloor$)

\end{itemize}

\section{Cicli Hamiltoniani --- Branch and Bound}
\subsection{K-Albero}
\begin{itemize}
    \setlength{\itemsep}{-3pt}
    \item Aggiungere i 2 archi più economici raggiungibili da K.
    \item Andando in ordine crescente di costo, aggiungere gli $n-2$ archi che
        non formano cicli tra di loro.
\end{itemize}
\subsection{Nodo più vicino}
\begin{itemize}
    \setlength{\itemsep}{-3pt}
    \item Algoritmo greedy: a partire dalla radice si sceglie sempre il
        percorso più economici per raggiungere il prossimo nodo.
        Ricordarsi di sommare il costo per tornare dall'ultimo nodo al primo.
\end{itemize}
\subsection{Branch and Bound}
\begin{itemize}
    \setlength{\itemsep}{-3pt}
    \item Partire dalla radice $P_{1,1}$, ogni figlio instanzia una variabile
        del tipo $x_{i,j}$ che se uguale a 1 indica che l'arco $(i,j)$ deve
        essere incluso nel K-Albero, se invece uguale a 0 l'arco non è
        utilizzabile.
    \item Se $V_I(P_{i,j}) \ge V_S(P)$, tagliare l'arco.
    \item Se $P_{i,j} = \{\}$, tagliare l'arco.
    \item Se $V_I(P_{i,j}) < V_S(P)$ e $V_I$ è generata da un ciclo
        hamiltoniano, tagliare l'arco e aggiornare $V_S(P)$.
\end{itemize}

\end{multicols}
\end{document}

